\documentclass[12pt]{article}
\usepackage{graphicx}
\usepackage{hyperref}
\usepackage{amsmath}
\usepackage{listings}
\usepackage{float}

\title{End-to-End Stock Risk Forecasting \& Recommendation System}
\author{Your Name Here \\ Course Info \\ Domain: Economics \& Finance}
\date{\today}

\begin{document}

\maketitle

\begin{abstract}
This report details the implementation of an end-to-end MLOps system designed to forecast stock returns, classify risk levels, and recommend similar stocks. The system leverages Alpha Vantage for real-time data, Scikit-Learn for modeling, Prefect for pipeline orchestration, and FastAPI for deployment.
\end{abstract}

\section{Introduction}
Stock market analysis requires robust pipelines to handle time-series data and deliver actionable insights. This project builds a comprehensive system to solve three key problems: forecasting short-term returns, classifying risk based on volatility, and identifying similar assets through clustering.

\section{Problem Statement}
Investors face challenges in quantifying risk and finding diversified assets. We aim to build a system that:
\begin{itemize}
    \item Predicts next-day returns ($y_{t+1}$).
    \item Classifies stocks into risk categories (Low, Medium, High).
    \item Recommends similar stocks based on market behavior.
\end{itemize}

\section{Methodology}

\subsection{Data Source}
We utilize the Alpha Vantage API (Time Series Daily). Data includes Open, High, Low, Close, and Volume.

\subsection{Feature Engineering}
Key features include:
\begin{itemize}
    \item Lagged Returns: $R_{t-1}, R_{t-2}, \dots$
    \item Rolling Volatility: $\sigma_t = \text{std}(R_{t-k} \dots R_t)$
    \item Moving Averages and Momentum.
\end{itemize}

\subsection{Machine Learning Models}
\subsubsection{Regression}
We use a Random Forest Regressor to minimize Root Mean Squared Error (RMSE):
\begin{equation}
    RMSE = \sqrt{\frac{1}{n}\sum_{i=1}^{n}(y_i - \hat{y}_i)^2}
\end{equation}

\subsubsection{Classification}
A Random Forest Classifier assigns risk labels based on volatility thresholds. We evaluate using Accuracy and F1-Score.

\subsubsection{Clustering \& Dimensionality Reduction}
We apply PCA to reduce dimensions, followed by K-Means clustering to group stocks.
\begin{equation}
    J = \sum_{j=1}^{k} \sum_{i \in C_j} ||x^{(i)} - \mu_j||^2
\end{equation}

\section{System Architecture}
The system is divided into an offline Training Pipeline (Prefect) and an online Inference Service (FastAPI). Models are versioned and stored in a registry.

% Placeholder for Architecture Diagram
% \begin{figure}[H]
% \centering
% \includegraphics[width=0.8\textwidth]{architecture_diagram.png}
% \caption{System Architecture developed in the project.}
% \label{fig:arch}
% \end{figure}

\section{Orchestration \& CI/CD}
Pipelines are managed by Prefect, ensuring reproducibility. GitHub Actions handles automated testing and Docker build verification on every push.

\section{Experiments \& Results}
(Fill in your results here after running the training pipeline)

\begin{table}[H]
\centering
\begin{tabular}{|l|l|}
\hline
Metric & Value \\
\hline
Regression RMSE & 0.XX \\
Classification Accuracy & 0.XX \\
\hline
\end{tabular}
\caption{Model Performance Metrics}
\end{table}

\section{Conclusion}
We successfully implemented an end-to-end MLOps solution. Future work includes integrating more tickers and advanced deep learning models (LSTM).

\end{document}
