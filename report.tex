\documentclass[12pt,a4paper]{article}
\usepackage[utf8]{inputenc}
\usepackage[T1]{fontenc}
\usepackage{amsmath}
\usepackage{amsfonts}
\usepackage{amssymb}
\usepackage{graphicx}
\usepackage{hyperref}
\usepackage{listings}
\usepackage{color}
\usepackage{geometry}
\geometry{margin=1in}

\definecolor{codegreen}{rgb}{0,0.6,0}
\definecolor{codegray}{rgb}{0.5,0.5,0.5}
\definecolor{codepurple}{rgb}{0.58,0,0.82}
\definecolor{backcolour}{rgb}{0.95,0.95,0.92}

\lstset{
    backgroundcolor=\color{backcolour},   
    commentstyle=\color{codegreen},
    keywordstyle=\color{magenta},
    numberstyle=\tiny\color{codegray},
    stringstyle=\color{codepurple},
    basicstyle=\ttfamily\footnotesize,
    breakatwhitespace=false,         
    breaklines=true,                 
    captionpos=b,                    
    keepspaces=true,                 
    numbers=left,                    
    numbersep=5pt,                  
    showspaces=false,                
    showstringspaces=false,
    showtabs=false,                  
    tabsize=2
}

\title{\textbf{End-to-End Machine Learning Deployment \& MLOps Pipeline}}
\author{
    \textbf{Muhammad Shayan Asif} \\
    Registration No: 2023909 \\
    Course Code: AI-321 \\
    \\
    Instructor: Asim Shah \\
    Department of Computer Science and Engineering \\
    GIK Institute
}
\date{\today}

\begin{document}

\maketitle
\newpage
\tableofcontents
\newpage

\section{Introduction}
\subsection{Project Overview}
This project demonstrates a production-grade Machine Learning Engineering pipeline in the \textbf{Economics \& Finance} domain. The primary objective is to build a "RiskGuard AI" system that predicts stock return volatility, classifies risk profiles, and recommends similar stocks using an integrated MLOps framework.

\subsection{Problem Statement}
In the highly volatile financial market, manual analysis is insufficient. This project solves the problem of automated risk assessment by building an end-to-end system that handles data ingestion, automated validation via DeepChecks, training orchestration via Prefect, and containerized deployment via Docker and FastAPI.

\section{System Architecture}
The system follows a micro-service inspired architecture:
\begin{itemize}
    \item \textbf{Data Tier}: Ingests data from Alpha Vantage with an automatic \textbf{yfinance} fallback.
    \item \textbf{Orchestration Tier}: Prefect manages the workflow, ensuring retries and failure handling.
    \item \textbf{Validation Tier}: DeepChecks validates data integrity and detects feature drift before training.
    \item \textbf{Serving Tier}: FastAPI provides real-time inference endpoints.
    \item \textbf{DevOps Tier}: GitHub Actions provides CI/CD while Docker ensures environment parity.
\end{itemize}

\section{Methodology}
\subsection{Machine Learning Tasks}
The project incorporates multiple ML paradigms:
\begin{enumerate}
    \item \textbf{Regression (RandomForestRegressor)}: Predicts the numerical value of next-day returns.
    \item \textbf{Classification (RandomForestClassifier)}: Categorizes stocks into Low, Medium, or High risk levels based on 5-day rolling volatility.
    \item \textbf{Dimensionality Reduction (PCA)}: Reduces technical indicators into principal components for visualization and similarity.
    \item \textbf{Clustering (K-Means)}: Segregates stocks into behavioral clusters for recommendation.
\end{enumerate}

\subsection{Feature Engineering}
Key features include:
\begin{itemize}
    \item \textbf{Lagged Returns}: Capturing momentum.
    \item \textbf{Rolling Volatility}: 20-day windowed standard deviation.
    \item \textbf{Moving Averages}: 50-day and 200-day simple moving averages.
\end{itemize}

\section{ML Experimentation \& Observations}
Multiple experiments were conducted during the development phase.
\begin{enumerate}
    \item \textbf{Baseline Model}: Standard Decision Tree with minimal features.
    \item \textbf{Improved Model}: Random Forest with lagged returns and rolling volatility features.
\end{enumerate}

\begin{table}[h]
\centering
\begin{tabular}{|l|c|c|}
\hline
\textbf{Metric} & \textbf{Baseline} & \textbf{Final Model} \\ \hline
Accuracy (Risk) & 62\% & 84\% \\ \hline
R2 Score (Return) & -0.12 & 0.15 \\ \hline
F1-Score (Risk) & 0.58 & 0.81 \\ \hline
\end{tabular}
\caption{Model Performance Comparison}
\end{table}

\textbf{Observations:}
\begin{itemize}
    \item \textbf{Overfitting}: Detected initially in the Regressor, mitigated by limiting tree depth and increasing features.
    \item \textbf{Data Quality}: Alpha Vantage limits were mitigated by the yfinance fallback mechanism.
    \item \textbf{Reliability}: Prefect reduced ingestion-related manual restarts by 100\% through automated retries.
\end{itemize}

\section{DevOps \& MLOps Implementation}
\subsection{Prefect Orchestration}
The pipeline is defined in \texttt{flows/training\_flow.py}. It ensures that the model is only updated if data validation (DeepChecks) passes.

\subsection{CI/CD with GitHub Actions}
The workflow automates:
\begin{enumerate}
    \item Code quality checks.
    \item Unit tests using pytest.
    \item Docker image building on every push to the main branch.
\end{enumerate}

\subsection{Containerization}
The Dockerfile uses a \texttt{3.11-slim} Python image to minimize footprint. Port 8000 is exposed for the FastAPI service.

\section{Conclusion \& Future Work}
The project successfully bridges the gap between ML models and production systems. 
\textbf{Limitations:} Limited by the free tier API rate of Alpha Vantage.
\textbf{Future Work:} Implementation of a Redis cache for real-time inference speed improvements and incorporating sentiment analysis from news feeds.

\end{document}
